\documentclass[a4paper,oneside]{article}

\usepackage{amsmath,amssymb} % amssymb has \lesssim, etc.
\usepackage{graphicx,pict2e,caption,multirow,cleveref}
\usepackage{natbib}
\usepackage{pstricks,pst-node,url}
%\usepackage[breaklinks]{hyperref}

\renewcommand{\familydefault}{\sfdefault}
\usepackage{sfmath}

\usepackage[draft,authormarkup=none,authormarkuptext=none]{changes}
\definechangesauthor[color=red]{m}
\setauthormarkup{} % Gets rid of superposed author id
    % Changes has #1 = author id, #2 = remark, so footnotes only contain remarks, no author id's.
%\setremarkmarkup{\footnote{#2}}

\usepackage[left=2.54cm,right=2.54cm,top=3cm,bottom=3cm]{geometry}
\def\CPP{{C\kern-.05em\raise.23ex\hbox{+\kern-.05em+}\hspace{4pt}}}

\title{Analysis of clusters}
\author{}
\date{}

\begin{document}
\maketitle

The first step in analysing clusters is to convert the raw bike trip data into matrices of correlations between all station pairs. This is done
with the ``{\bf bikes\_get\_r2\_mats.cc}'' program. The London version of this program produces the following output files:
\begin{enumerate}
    \item{ntrips.txt --- An [nstations$\times$nstations] matrix of numbers of trips per day}
    \item{results\_list.txt --- A vector list of ALL results data; that is, a list containing, for each pair of stations, the station lats
        lons, distance between, and correlations between rides ridden both from and to. Length of vector is thus [nstations$\times$nstations].
        This file is used to subsequently calculate k-values describing Gaussian declines of correlation with distance from each station. (See
        R routine ``calc.kvals'' below.)}
    \item{results\_st\_latlons.txt --- Just what it says, in a matrix of length nstations}
    \item{results\_r2\_from.txt, .\_r2\_to.txt - The actual inter-station correlation matrices.}
\end{enumerate}
The DC version just produces the correlation matrices, along with ``results\_st\_latlons\_DC.txt.'' In both cases, the correlation (r2) matrices
form the input for the actual analyses of clusters, which initially uses routines from ``plotresults.R,'' which is the main R analysis script.
It includes the following functions:
\begin{enumerate}
    \item{plotresults --- Plots visual clustering schemes for a given number of clusters. Relies upon ``drawlondon\_script.r.'' This is the
        routine used to identify strange points and relocate them into more appropriate clusters using the routines below.}
    \item{plotresults.osm} --- Dumps a hi-resolution version to file with OSM buliding polygons shaded by cluster membership. This routine
        relies on the following two sub-routines
    \item{add.polys} --- Adds polygons to plot
    \item{poly.in.hull} --- Identifies which polygons lie within a given hull describing a cluster.
    \item{swap.clust.memb --- Identifies a point within a cluster based upon whether it is that farthest point to the [N,S,E,W], and relabels
        its cluster ID.}
    \item{renumber.clusters --- Allows entire clusters to be renumbered, both so colours and numbering schemes align better, and so degrees of
        cluster overlap can simply be measured by aligning clusters according to their consistent numbering schemes.}
    \item{get.clusters --- The primary hierarchical clustering scheme, which also automatically renumbers clusters based on maximal degrees of
        overlap}
    \item{get.clusters2 --- Just strips out the actual clustering bit from above, without renumbering clusters, and dumps the ``merge'' field of
        the hclust output to a file, ``hclust\_merges.txt.''}
    \item{get.clusters3 --- Does the manual renumbering identified through the main plotresults routines, applies it for clustering schemes of
        (currently) up to 20 clusters, and dumps the resultant clustering schemes in 2 matrices of [nstations, 20], called
        ``clust\_from\_members.txt'' and ``.\_to\_members.txt''.}
    \item{get.clusters.kmeans --- Dumps cluster IDs in two [nstations$\times$max\_clusters] arrays called ``kclust\_from\_members.txt'' and
        ``.\_to\_members.txt.''}
    \item{get.members --- Analyses the merge history produced by hclust to strip out a given number of clusters. It differs from the R internal
        ``cutree'' routine in preventing clusters from having less than 3 members by automatically merging such clusters with the geographically
        closest larger clusters.}
    \item{calc.kvals --- Calculates the k-values describing Gaussian decays in correlations from each station, and dumps them to an output file,
        along with station lats \& lons.}
\end{enumerate}
There is then also a DC version of plotresults, but the actual plotresults routines has not really been implemented, because DC analyses work
even better than the London ones without any cleaning up. The DC one also does not yet have an OSM plotting option.

The primary function of both plotresults routines (apart from allowing visual inspection of London clusters) is to use the correlation matrices
to produce files detailing cluster membership. For London, this is done by get.clusters3, which produces the final
``clust\_(from,to)\_members.txt'' files. The ``get.clusters.kmeans'' routine produces the corresponding values for comparison. The equivalents
for DC are simply get.clusters and get.clusters.kmeans.

{\bf NOTE ON K-MEANS CLUSTERING} k-means clusters are intitiated with random selection. The parameter nstart (default = 1) can be increased to
repeat the clustering with a number of random starts, with the convergent estimate returned. Using nstart = 5 seems to give evidently more
reliable clustering procedures than nstart = 1.

The cluster membership files are then used to measure total distances ridden within each form of cluster, using 
the \CPP program, ``clusters\_actual.cc,'' and the DC equivalent. Clustering can be done using either:
\begin{enumerate}
    \item{The standard hclust output}
    \item{(London only) The hclust output with manual corrections; or}
    \item{k-means clustering}
\end{enumerate}
These \CPP programs produce output files which contain 2 columns of [number of clusters, total distance ridden {\it within} clusters]. The
equivalent program, ``clusters\_neutral.cc'' calcualtes the same values through averaging over a given number of aritrarily assigned, yet
spatially contiguous, clusters. The present results data have been averaged over 100 repeats for each number of clusters. 

These observed values of distances ridden are then statistically combined and compared in the routine ``cluster\_significance.R,'' which
simultaneously analyses both London and DC results. The observed distances ridden are converted to T-statistics using the neutral values. These
T-statistics are then analysed to discern the positions of all peaks, and thus the M values that best reproduces the observed series. (From
peak $=1+N*(M-1)$). This is done using the align.peaks2 routine, with align.peaks just minimising numbers of missing observations, and kept for
archival reasons only. (May in fact be deleted.)

London produces $M\approx 4$, which is also the value produced from a random series, so that finding on its own can not be
interpreted to mean anything. What is thus needed is also to measure the extent to which peaks really are peaks. I thus devised a routine to
measure how distinct the peaks were, by rescaling the data using a non-linear routine to estimate upper and lower bounds, and rescaling these
between -1 and 1. Peak distinctness is then the peak height minus the average of the two adjacent points. This value has a maximum of 2.

The routine calc.pm calculates the probability distribution of peak distinction using only series fitting a given $M$-value. For a given value
of peak distinction, it returns the probability. The probabilities of each series of peaks for that given $M$-value can thus be calculated, and
the probability of the entire series finally calculated as the product of all individual values. The DC peaks correspond to $M=5.8$, which alone
almost never arises from random data, and so they are entirely significant regardless. The London data are also revealed to be entirely
significant, thus finally demonstrating the validity of the analysis.

This cluster\_significance.R script also has an extra routine, ``clust.stats,'' which calculates the internal statistics of the
clustering algorithm. This shows that these internal statistics can also discern peaks at roughly the same places as with the relative
intra-cluster distances above. These peaks are nevertheless not as distinct, and this routine exists for reference only.

A final routine, ``calc.b,'' takes a given exponent b in $\binom{N}{k} \div k ^ b$ and produces the corresponding $M$-value. This enables the
translation of observed $M$-values into associated exponents.

\end{document}
